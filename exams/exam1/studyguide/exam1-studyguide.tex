\documentclass[]{article}

\begin{document}
\begin{center}
\Large Exam 1: Study guide
\end{center}

The exam is open internet. However, I strongly recommend you to prepare for the exam and not bank on the fact you can search online. I recommend preparing a summary sheet of 8.5 x 11 inch of your own handwritten notes - this is a good learning exercise. The exam questions will be determined such that they satisfy a subset of the objectives listed here.

Exam 1 will cover:
\begin{itemize}
	\item Lecture handouts \# 1, \#2
	\item Homeworks 1 and 2 (Variables, Iterations and Control flow)
	\item Labs 00 through 01c (Variables, Iterations and Control flow, nested loops and plots)
\end{itemize}

To perform successfully on Exam I, you should be able to:

\begin{enumerate}
	\item Determine data types (\verb|bool|, \verb|int|, \verb|float|, \verb|string|) of different operations (for e.g, $*, +, -, /, \%, //$)
	\item Write simple programs using variables and evaluate the value bound to a variable(s) at different step(s) in the code.
	\item Understand and evaluate the order of precedence in a given statement(s).
	\item Identify syntax, semantic and static-semantic errors in the code. It is not required to identify the exact type of error in the code, but the location of an error(s) in the code.
	\item Identify and fix incorrect code either using the output error message, verification result or logic.
	\item Understand and develop logic / algorithms / code that involve iterations (single and nested \verb|for|, \verb|for - else| and \verb|range|) and control flow (\verb|if|, \verb|elif|, \verb|else|, \verb|break| and \verb|continue|).
	\item Understand the output of a given code or expression (for e.g., \verb|range()|)
	\item Write bisection approach to solve non-linear equations and list drawbacks of the bisection approach.

\end{enumerate}

You won't be required to write lengthy code (more than 30 lines). I will not penalise for obvious typos and syntax errors in your code (for e.g., missing \verb|:| at the end of control flow statements), unless that is what is tested.
\end{document}
