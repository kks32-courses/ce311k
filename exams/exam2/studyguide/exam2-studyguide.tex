\documentclass[]{article}

\begin{document}
\begin{center}
\Large Exam 2: Study guide
\end{center}

Exam 2 for CE 311K is a one-hour open-internet exam. The exam questions will be determined such that they satisfy a subset of the objectives listed here.

Exam 1 will cover:
\begin{itemize}
	\item Errors and functions
	\item Homeworks 02 through 03 (Errors, functions)
	\item Labs 02 to 03 (Errors, functions)
\end{itemize}

To perform successfully on Exam II, you should be able to:

\begin{enumerate}
	\item Determine data types and outputs during casting operations. Please note only Python standard data types (\verb|str|, \verb|int|, \verb|float|, \verb|bool|) will be covered. Numpy data types are not included (for e.g., \verb|np.float16| and others)
	\item Evaluate relative and absolute error(s) for a given program and develop  a suitable termination criteria (\verb|break|) or (\verb|return| in case of a function).
	\item Define function with arguments (including default) and multiple return types for a given problem and call (use) the functions in a Python code.
	\item Rewrite a given program using functions to make re-use of code as much as possible.
	\item Identify and fix errors in passing function arguments and return types.
	\item Evaluate the output of a given function(s).
	\item Evaluate the value of different variables within and outside the function (scoping)
%	\item Develop Python code that use, index, manipulate and search (\verb|in| and \verb|not in|) lists.
%	\item Iterating through a list using indexing and \verb|in| operations.
%	\item Write a simple list comprehensions with a filter for a given list and a condition.
%	\item Deduce the value of a variable after trying to modify a list item and a tuple using an index or a key.
%	\item Use of dictionary is \textbf{not} part of the exam.
%	\item Develop Taylor series approximation for non-polynomial functions for single variable functions. Write a Python code to solve for the Taylor approximation with relative errors.
%	\item Develop Newton-Raphson code to find the root of a function. Compute the tolerance error at each iteration.
%
%
\end{enumerate}

You won't be required to write lengthy code (more than 30 lines). I will not penalise for obvious typos and syntax errors in your code (for e.g., missing \verb|:| at the end of function definitions), unless that is what is tested.
\end{document}
