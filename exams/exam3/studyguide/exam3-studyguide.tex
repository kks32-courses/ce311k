\documentclass[]{article}

\begin{document}
\begin{center}
\Large Exam 3: Study guide
\end{center}

The exam is open internet. However, I strongly recommend you to prepare for the exam and not bank on the fact you can search online. I recommend preparing a summary sheet of 8.5 x 11 inch of your own handwritten notes - this is a good learning exercise. The exam questions will be determined such that they satisfy a subset of the objectives listed here.

Exam 3 will cover:
\begin{itemize}
	\item Data structures, Taylor series and Newton Raphson
	\item Assignments 04 (Taylor series and Newton Raphson)
	\item Labs 04 and 05 (Data Structures, Taylor series and Newton Raphson)
\end{itemize}

To perform successfully on Exam III, you should be able to:

\begin{enumerate}
	\item Develop Python code that use, index, manipulate and search (\verb|in| and \verb|not in|) lists.
	\item Iterating through a list using indexing and \verb|in| operations.
	\item Deduce the value of a variable after trying to modify a list item and a tuple using an index or a key.
	\item Use of dictionary is \textbf{not} part of the exam.
	\item Develop Taylor series approximation for non-polynomial functions for single variable functions. Write a Python code to solve for the Taylor approximation with relative errors.
	\item Develop Newton-Raphson code to find the root of a function. Compute the tolerance error at each iteration.

\end{enumerate}

You won't be required to write lengthy code (more than 30 lines). I will not penalise for obvious typos and syntax errors in your code (for e.g., missing \verb|:| at the end of function definitions), unless that is what is tested.
\end{document}
